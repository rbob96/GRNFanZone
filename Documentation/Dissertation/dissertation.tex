\documentclass{l3proj}

\usepackage{float}

\begin{document}
\title{Global Rugby Network FanZone (Web)}
\author{Ruxandra Bob \\
		Marios Constantinou \\
        Daniel Juranec \\
        Arnas Kapustinskas \\
        Andrew McCluskey}
\date{10 February 2017}
\maketitle
\begin{abstract}
Report on Team Project 3 coursework for Group V. The GRN FanZone is a
 fan engagement web application developed to interface with the Global Rugby
 Network's online platform. The project tried to utilise
 emerging technologies ranging from Angular to Firebase, in addition to
 cutting-edge methodologies, such as Behaviour Driven Development and Agile
 project management. This report covers some of the considerations taken
 during development and explains some of the benefits and disadvantages
 of the way the development team operated. The report will also cover the lessons
 the team learned over the course of the six months.
\end{abstract}
%% Comment out this line if you do not wish to give consent for your
%% work to be distributed in electronic format.
\educationalconsent
%\newpage
%\tableofcontents
\newpage
%==============================================================================
\section{Introduction} %1-2 pages

This paper presents a case study of the project and software development process
 of Group V, which consists of five Software Engineering students at the
 University of Glasgow, as part of the University's Team Project 3 course.

Our customers for the project were Global Rugby Network (GRN), who tasked us
 with creating a web application that would allow professional and amateur
 rugby clubs, teams, and players to interact with rugby fans. This social
 platform would then be integrated with the software that GRN already provides
 to rugby clubs. The end result is GRN FanZone, a fully functioning web application
 that allows people to follow their favourite clubs, see the latest posts
 made by them, find rugby matches happening near them, and much more.

The purpose of this document is to reflect on the achievements and challenges
 that arose during the development of this project. It also explores the concepts
 and methodologies touched upon in the concurrently running Professional Software
 Development (PSD) course, how we decided to implement them, and the effectiveness
 of our implementation. The document is structured in a way that will start from
 a customer-facing perspective, working through to the backend and then finally
 our project management techniques.

The rest of the case study is structured as follows.

\nameref{sec:background} presents the background of the case study
 discussed, describing the customer and project context, aims and
 objectives and project state at the time of writing.

\nameref{sec:frontend} talks about the key technology used for the
 frontend implementation of the project. It contains an overview of the
 technology and justification for its usage in the project.

\nameref{sec:backend} covers the key technologies used for the
 backend development of the project. This section will also describe
 some of the difficulties faced by the team in this area.

\nameref{sec:testing} will reflect on the difficulties faced when
 writing tests for the project in Karma Jasmine, a test-runner and framework
 combination recommended for use with our frontend technology. It will continue
 to give examples of how these testing practices benefitted the team.

\nameref{sec:cicd} discusses the Continuous Integration and Continuous
 Deployment techniques that were used in order to ensure higher quality
 of code and to continuously deliver new implementations. It also outlines
 challenges the team faced when using these techniques.

\nameref{sec:planning} will show how the project was planned out,
 what pitfalls we faced, and how we overcame them. It will also lay out
 some of the lessons we learned about planning a large project.

\nameref{sec:agile} explains some of the agile practices we used in our 
 project.

\nameref{sec:changemgmt} describes how the team used their VCS and managed the 
 project in day-to-day life. This section will describe the lifecycle of an issue ticket in
 depth.

\nameref{sec:conclusion} is the conclusion, where we evaluate our 
 performance over the whole process, what we have learned, and how 
 this experience can benefit us in following professional software projects.


%==============================================================================
\section{Case Study Background}
\label{sec:background}

\subsection{Client Background}

The Global Rugby Network (GRN), is a free team management platform
 for rugby teams around the world. Their aim is to provide high-quality
 team and performance management tools to amateur and semi-professional 
 rugby teams worldwide. 

The team consists of 7 members, all of whom have played rugby at a high level. 
 As such, each of them understands the problem domain in great depth. After years in
 professional rugby, and spending time talking to amateur and semi-professional coaches, 
 Dave Millard, the CEO discovered "amateur teams and emerging rugby nations... find it 
 hardest to access professional team management, coaching and performance software." 
 GRN's product combines video-tagging, team management and team communication to help 
 fill this gap.


\subsection{Initial Objectives}
As mentioned above, our client's primary objective is to provide a platform that
 enables coaches to manage their teams.  The client noticed that there was a lack of
 fan engagement opportunities on their platform. As a result, our project motivation
 and initial objective was to build an online social platform for rugby fans, who could
 be kept up to date with official information from teams.

GRN FanZone was developed as a web application and aims to provide users with a way of
 interacting with their favourite clubs, teams and players, inside a user-friendly
 environment. The two initial target audiences included the profile-owners and the
 profile-followers.

Profile-followers are the users who will be registering accounts with GRN FanZone to
 stay up-to-date with rugby news. As an initial step, followers sign in using Google or
 Facebook accounts. After this, users are presented with a list of clubs, teams and players
 that they can choose to follow. Another requirement is letting profile-followers "unfollow"
 a currently followed entity. There is also the opportunity to get information about
 upcoming fixtures.

Profile-owners represent the group of clubs, teams and players who have registered with
 GRN. These entities will not access the web application directly, but use their GRN account
 to provide information, which is subsequently pushed to the GRN FanZone and then presented
 to the profile-followers.
 
A club is the highest level entity in the hierarchy. A club can be the parent of many different
 teams - for example; a club could have a juniors team for under 21-year-olds, a women's team and
 a men's team. These teams will often have the same location. A team consists of many players. A
 player can play for more than one team - Zander Fagerson plays for both the Scotland
 team at an international level and the Glasgow Warriors at a professional level.
 
These entities will interact such that the profile-owners will be able to disseminate
 information to their profile-followers. In the future, the platform aims to include 
 media-based posts such as video and audio, but as this introduces additional complexity
 to the platform, GRN asked that we focus on textual posts for now. This allows the post system
 to be extended later.

 
Breaking the objectives down into more distinct areas, we can see three primary areas for
 the platform to cover:

\begin{itemize}
\item[1] Club/Team/Player news in the form of posts
\item[2] Club/Team/Player information in the form of their profile pages
\item[3] Fixture information from fixture detail pages 
\end{itemize}

Sharing news as posts is important to improve fan engagement as it allows
 profile-followers to be kept up to date with information that is generated quickly,
 and it is relevant only for a short period of time. The post system reflects 
 that, as each post is gradually hidden under newer posts displaying fresher 
 information.

The profile pages for each profile-owner are designed to allow static 
 information to be easily found. This includes genders, locations, age-groups and 
 so on. There are different sets of information for different entities, so these
 have to be reflected by the different pages.
 
Fixtures are more interesting, as they are a crossover - they will be over soon, 
 but we also have to include a lot of team information on them, as well as the
 other information, such as the time of kickoff, or the score. We opted to make 
 this a separate page, which is linked to from a team's page.
 
As the product is intended to be used across the world, internationalisation
 support was favoured by GRN when we suggested the idea. By easily integrating
 other languages, we can open up the app to a much wider range of people - for
 example, rugby is also popular in Italy, France and Japan, to name but a few.

In general terms, our aim was to use technology to encourage engagement amongst the
 rugby community - ranging from fans to sponsors.

\subsection{Current State of Product}
The GRN FanZone is currently deployed to \url{http://grnfanzone.firebaseapp.com}.
 The product allows social sign in using facebook or google accounts. Following
 login, the user is presented with a screen that allows them to follow clubs,
 teams or players. Once a user has done this and reloads the dashboard, they
 are presented with a news feed of posts. This feed uses infinite scrolling to 
 reduce the amount loaded on launch but not impact the user experience.
 
Profile pages for clubs, teams and players are also all implemented. They display
 the key information for each of their respective entities. They also list 
 recent posts that entity has made and give the option to follow or unfollow 
 the entity.
 
Fixture pages show the kickoff time, the two teams playing and the score (if the
 game has kicked off already). The page also shows the location of the game,
 and displays it on a map, powered by the Google Maps API.
 
Regarding internationalisation support, we managed translations for: Romanian,
 Greek, Lithuanian, Russian, Japanese, Italian, Spanish and Chinese. These 
 translations are mostly complete - Chinese and Spanish both have some phrases
 not translated. This was due to the fact that we approached speakers of these 
 languages before all of the phrases were finalised. 


%==============================================================================
\section{Frontend Technology} %1-3 pages
\label{sec:frontend}

At the start of the project, GRN chose to give us a set of technologies they
 wanted us to use. These included defining Angular 2 as our frontend framework.
 Angular is a framework maintained by Google, which has been gaining popularity
 over the last eight years\cite{angularjsoverview}. The framework uses typescript,
 a type safe flavour of javascript. Angular focuses on being portable, high performance
 and easy to write\cite{angular_features}. Many of the contributors to Angular are
 from Google, but the project is open source under the MIT license, and anyone can
 commit to it\cite{angularjsoverview}. Angular is built on top of NodeJS, providing
 easy access to a wide range of external components.

Angular embeds code in HTML and uses a "controller" to define component behaviour. This
 "component" object is the basis of most Angular development. Angular also tries
 to minimise the logic in controllers, by either abstracting it to other components
 or building "services" - shared libraries of code snippets. The final part of this
 structure is a "module" which coordinates resource injection into controllers from one
 file. A project may have more than one module for its subsections, but we stuck to
 one file. We made use of several services, and many components.

 % Pros and cons of ng-cli
Angular provides a CLI that makes it easy to start projects and provides
 templates for components, services and modules. This provides the \texttt{ng init}
 command, which prepares a skeleton project in a matter of minutes.
 The basis of the project was generated in under a minute, which set up
 testing frameworks, a basic starter app and a \texttt{package.json} file for NPM.
 When generating a new component, a HTML template, CSS styles file, karma
 jasmine test file as well as the typescript controller. These were populated
 with template code to allow easy starts. This helped ensure we could get to
 the business logic of the component or service quickly, without wasting time.

  \begin{figure}[H]
\begin{center}
\includegraphics[width=7cm]{figures/frontend_components}
\end{center}
\caption{A selection of components used in the GRNFanzone}
\label{fig:frontend_components}
\end{figure}


 % Why we liked it (easy to maintain, works out the box, good documentation)
Angular 2 gave us a huge advantage when working on this project. Components provide
 an easy way of separating concerns and reducing code duplication. Generating new
 code is easy with the CLI, and entire projects can be started and running
 in hours. The documentation for Angular is also excellent. Maintained by google,
 it is kept up-to-date with the regular releases Angular receives.

 % why we disliked it (steep learning curve)
The most significant barrier to using Angular was our own inexperience.
 Thus far, our formal education has not involved developing web apps in
 Javascript based frameworks. Beyond this, Typescript was new to every
 member of the team. This meant that there was a steep learning curve for
 the team to deal with and adapt to. This caused some issues early on
 before we had a good grasp of how to use the actual features Angular
 provides.

 % How easy it was to test
One of Angular's benefits is that its developers have carefully
 considered testing. Software testing has long been an important
 aspect of software development, and this has only become more true
 in recent years \cite{tuteja2012testing}. The Angular CLI generates
 a functional karma file, which has been configured for Angular. The
 Angular CLI also provides an interface for running tests. Parameters
 range from single/constant runs of the test suite to enabling code
 coverage reporting. In spite of this, it was found to be very
 difficult to mock our services which used angularfire to generate
 useful tests (please see section \ref{sec:testing} for more on this).

\begin{figure}[H]
\begin{center}
\includegraphics[width=11cm]{figures/frontend_ng_cli}
\end{center}
\caption{Using the Angular CLI to run tests and linting}
\label{fig:frontend_ng_cli}
\end{figure}


%==============================================================================
\section{Backend Technology Considerations} %1-3 pages
\label{sec:backend}

As with the frontend, our clients defined what technology they wanted the team
 to use for the backend of the project before we started development. We were
 tasked to use Firebase, which is a mobile and web application platform that
 was acquired by Google in 2014. Its initial product was a NoSQL real-time database,
 however, since then it has added a large range of features, including static
 hosting services, authentication and storage.

We started off by registering our project "GRNFanZone" on the Firebase website,
 which provided us access to the Firebase console, as well as an API key to
 initialise and access the database. Afterwards, we followed the documentation
 to implement authentication using Facebook and Google, which was an initial
 requirement for the project. Finally, the team used Firebase's CLI to deploy
 the initial build using the provided static hosting service. This process
 was later refined and automated as part of our continuous deployment strategy
 (discussed in section \ref{sec:cicd}). Overall, the ease with which Firebase
 allowed these features to be set up and implemented benefitted us greatly,
 as it allowed us to concentrate on other parts of the project.

The main benefit of Firebase for the project was the real-time cloud-hosted
 NoSQL database. All of the data in the database is defined using Javascript
 Object Notation (JSON). The database can be edited by using the editor
 in the online Firebase console, or by creating a JSON file and either deploying
 it using the CLI or importing it in the console. We used Globally Unique
 Identifiers (GUID) for main entity IDs, and divided our data into six main categories:
\begin{itemize}
\item
\textit{Users}, where every user would contain some personal information
 from social authentication, as well as clubs, teams and players that user
 follows.

\item
\textit{Posts}, containing the text and title of the post, the post
 author ID and some information about him, the comments and likes for
 that post, and a timestamp.

\item
\textit{Fixtures}, which contains the home and away team IDs and names,
 timestamps of fixture creation and kickoff, location, and the score if
 the fixture has already happened.

\item
\textit{Clubs}, where each club has their crest, name, description,
location, and the posts made by that club.

\item
\textit{Teams}, which, alongside the same information as
clubs, contains what club the team belongs to, the team's age group and
type, alongside posts made by the team.

\item
\textit{Players}, that has every players' name, age, bio, location and
 profile picture, as well as what team the player plays for and posts
 by them.

\end{itemize}
The angularfire2 package that was used to interact with the backend used
 observables to provide data; information was updated on the frontend in
 real-time, even without refreshing the app. As the purpose of the web
 application was to provide a social platform for rugby fans, all connected
 users being synced to the database made for a much natural and interactive
 user experience.

\begin{figure}[H]
\begin{center}
\includegraphics[width=11cm]{figures/backend_dbexample}
\end{center}
\caption{A snippet of the JSON used as the example database for GRNFanZone}
\label{fig:backend_dbexample}
\end{figure}


Firebase did not, however, come without its drawbacks. As usage of NoSQL
 databases has only recently become more prevalent, none of the team members
 had previously had experience of using or modelling data for one. Perhaps
 the biggest issue that arose while getting acquainted with Firebase was
 changing our relational database data modelling mentality to one that would work
 well with a NoSQL database. For example, duplicated data, which is considered
 undesirable in an SQL database, was necessary for us to be able to retrieve
 information about posts efficiently. It also meant that we had more limited
 functionality when it came to querying the database.

Another big issue was identified later on in the project. We used the
 Angularfire2 module to interface between Firebase and Angular2, which meant
 that all of our calls to the database had to be in the form of subscriptions
 to certain keys. As the application grew in complexity, more and more
 subscriptions had to be made, due to which there arose some race conditions
 concerning the load order of the data and issues where retrieved data would
 be needlessly reloaded. Fixing these problems was very frustrating and took
 up a lot of time that could have been spent implementing new features.

%==============================================================================
\section{Testing Considerations} %1-3 pages
\label{sec:testing}

% Describe BDD
Behaviour Driven Development is a development methodology where the development team
 first describes use cases for a feature, then writes tests for it, and lastly
 develops the code for the feature. The team decided to use BDD as it comes
 recommended by the Agile Alliance \cite{agilealliance_bdd}. Another reason
 is that the angular CLI sets up a Karma Jasmine testing set up by default.
 BDD is credited with helping to develop easily documented code - for example,
 by making test cases into natural language sentences (see figure
 \ref{fig:testing_failed_test}), it becomes easier for human developers to
 understand what it tests, and why it is being tested \cite{north2006bdd}.

 \begin{figure}
\begin{center}
\includegraphics[width=11cm]{figures/testing_failed_test}
\end{center}
\caption{A failed test gives a human-readable failure message}
\label{fig:testing_failed_test}
\end{figure}

% Describe Karma and Jasmine testing with Angular2
The Angular 2 CLI sets up a Karma Jasmine configuration to run tests on
 the typescript in the project. Karma is a test runner for javascript
 that is platform agnostic and allows testing on a range of different devices
 and browsers\cite{jina2013javascript}. Jasmine is a testing framework
 that aims to be as readable as possible - one of the key features of
 the BDD approach we wanted to adopt. Initially, we tried to use Mocha,
 another testing framework which is targeted at Test Driven Development,
 however integrating it was difficult and was taking too long to set up,
 so we decided to move from TDD to BDD and just use Jasmine. One of the
 benefits this brought was the wealth of documentation and examples
 across the internet of using Karma with Jasmine on Angular code.

% Describe how it was done well
% mention test coverage
By the end of the project, a number of tests had been incorporated into the
 project. These tried to cover edge cases and a variety of conditions the GRN FanZone
 could face in the wild. Our code coverage - a metric which enumerates what
 proportion of the code is tested - tended to stay fairly high. We have exact
 figures due to our Continuous Integration, which ran the tests every time a
 commit was pushed(section \ref{sec:cicd}), though we did not place any constraints
 on a minimum coverage value. Typically, the coverage ranged from 60\% to 80\%.
 Towards the end of the project, the value was just over 75\%. This was
 representative of just under 40 tests written. To increase transparency
 across the project, we displayed values for our test coverage on \texttt{dev} and
 \texttt{master} in the readme, using icons provided by GitLab (see figure
 \ref{fig:cicd_coverage_buttons}).

 The code coverage metrics also provided a file-by-file breakdown of where
 coverage was high or low. This proved to be helpful when choosing
 where to focus our effort - there's no point in writing tests for
 well-tested code, it makes much more sense to write new tests for the
 code that is less well tested. See figure \ref{fig:testing_coverage_metrics}
 for an example of the HTML display.
 
% Difficulties we faced
% reference common pitfall from agile alliance of being a big jump from tdd and hard for noobs
% Struggles with angularfire2
% struggles with @input components
Testing was not always easy. For example, getting into a habit of writing tests
 proved difficult for the team, showing that working in a development team on a
 complicated project is very different to working on individual projects. The
 Agile Alliance warns that "BDD requires familiarity with a greater range of
 concepts than TDD does, and it seems difficult to recommend a novice programmer
 should first learn BDD without prior exposure to TDD concepts"
 \cite{agilealliance_bdd}. This turned out to be a significant difficulty for
 our team, demonstrating to us first hand that it is important to choose where
 in the project a team should focus their learning time and that too much new technology
 can stretch a team to the point where the effort spent to learn new things
 overtakes that spent on features.

\begin{figure}[H]
\begin{center}
\includegraphics[width=11cm]{figures/testing_coverage_metrics}
\end{center}
\caption{The HTML report on code coverage}
\label{fig:testing_coverage_metrics}
\end{figure}


Another struggle was getting tests to run. Due to the interconnectivity of
 Angular components, a complex data flow has to be replicated. For example,
 components that use the \texttt{@input()} tag to receive data don't have a
 trivial way of faking this. This means that the data flow there becomes
 difficult to test accurately. Had this not been such an issue, the post
 component would have been much easier to test - currently, it has the worst
 code coverage in the project.

There were also issues when testing backend connections. While it is
 allegedly possible, the team did not find a way of mocking Angularfire2, the
 service which provides an interface between Angular and Firebase. This meant
 that tests would affect the production database, something which could not be
 allowed. As a result, calls made to Angularfire were not tested, despite the
 fact that they constituted a significant amount of our business logic.

Overall, the team's testing methodologies were too optimistic. We assumed that
 we could make a large jump to an unfamiliar methodology, while also learning
 to write new tests. Another of the mistakes we made was putting testing aside
 and instead trying to get features out. This resulted in a code base that did
 not have the test coverage we desired, but also gave us a large task in writing
 the tests after the fact. This proved more difficult than writing them as we
 went along.

%==============================================================================
\section{Continuous Integration and Deployment} %1-3 pages
\label{sec:cicd}

The project used a multitude of Continuous Integration (CI) and Continuous
 Deployment (sometimes Delivery) (CD) techniques. A CI server's purpose "is to check the code
 repository for changes, check out the code if it spots any [changes], and run a
 list of commands to trigger the build."\cite{meyer2014continuous} A build is "ideally more than just
 compiling - it should also include a thorough test suite to help verify that the code
 still works with every change."\cite{meyer2014continuous} This gives a development
 team a quick, automated way of checking their code works, follows a style guide and
 doesn't break any other work.

One of the key concepts of CI is often phrased as: "Commit Daily,
 Commit Often"\cite{meyer2014continuous}. For our project, this was sometimes a struggle.
 This was due to a small number of factors, which boiled down to: "We don't work on the project
 every day", and "I'm not used to git". There was little we could do to remedy the former issue -
 all we could do was commit often \textit{while we worked on the project}. The Gitflow
 branching system we used in the VCS (see section \ref{sec:planning}) was unfamiliar to several
 members of the team, and it took time for everyone to become accustomed to the system. As the
 project progressed, however, more builds were made, more commits pushed, and more bugs found.

Another hurdle at which we fell was getting into the habit of writing tests for our code. I shall
 mention this briefly here, but for more details, please see section \ref{sec:testing}. In Fowler's
 2006 paper on CI, he says: "Imperfect tests, run frequently, are much better than perfect tests
 that are never written at all."\cite{fowler2006continuous}

Continuous Deployment is the practice of continually deploying working builds to production
 as often as possible. It adheres to the Agile principles of:
 \begin{itemize}
 \item
 Our highest priority is to satisfy the customer
 through early and continuous delivery
 of valuable software. \cite{agileprinciples}
 \item
 Deliver working software frequently, from a
 couple of weeks to a couple of months, with a
 preference to the shorter timescale. \cite{agileprinciples}
 \end{itemize}
 In our project, as soon as a new feature is merged into the \texttt{dev} branch,
 tests are run, and then the changes are deployed to a staging server, hosted by firebase. Similarly,
 as soon as \texttt{dev} is merged into \texttt{master}, \texttt{master} is pushed to our production server, also hosted by Firebase.
 The \texttt{dev} branch was typically merged into \texttt{master} once a week, allowing time for any changes to be made
 to features that weren't quite perfect, and to iron out any bugs that were found after time in \texttt{dev}.

Our CI and CD was operated using GitLab's integrated CI system. This uses Docker to
 run a set of instructions defined in the 'gitlab-ci.yml' file.  Instructions are
 separated into tasks. Tasks belong to stages - in our project, these were
 "test" and "deploy". The tasks are run as builds within a pipeline. The Docker instances were
 in some cases hosted for free by GitLab (sponsored by a cloud company). In other cases,
 builds were run on team computers. A common upper limit for build time is quoted as
 10 minutes\cite{fowler2006continuous} - ours typically ranged from 5-15 minutes, with some exceptions that were
 typically waiting on the GitLab CI runners to free up. The single largest time-drain was
 running \texttt{npm install} every time docker span up a new instance. Running builds
 typically took under a minute after this.

 \begin{figure}[H]
\begin{center}
\includegraphics[width=9cm]{figures/cicd_build_duration}
\end{center}
\caption{Build Duration for last 30 commits from 21:28 on 19/03/2017}
\label{fig:cicd_build_duration}
\end{figure}


% Describe Linting task
The first task used by the project's "test" stage ran a series of lint checks over the
 project's source code. Linting involves performing static analysis on code to detect bugs
 or violations of a style guide. These kinds of checks are also performed by compilers and
 so on. These issues can range from missing semicolons to using a mix of
 double and single quotes, to whether a function is never called. The task tested CSS,
 JSON, Typescript, Javascript,  HTML and LESS. While this was often annoying, these tests
 did help maintain a higher quality of code in the codebase. As our policy was to not allow a
 merge to \texttt{dev} take place if a branch was not passing tests, we had a method of
 enforcing that the standards we defined were upheld.

% Describe testing task
The second task ran the project's tests. Again, this task had to complete successfully 
 for a branch to be merged into \texttt{dev}. This stage also generated coverage reports which
 were used to give the team an indication of how well our code was tested.

 \begin{figure}[H]
\begin{center}
\includegraphics[width=9cm]{figures/cicd_failed_pipeline}
\end{center}
\caption{Example of a failed pipeline in GitLab}
\label{fig:cicd_failed_pipeline}
\end{figure}


% Considered combining test tasks
Merging these two tasks was considered, but left aside for now. The tasks take 5-10 minutes
 each to run and are run in parallel. The downside of this separation is the fact that \texttt{npm install}
 is run twice. However, by leaving the tasks separate, we get a quicker, clearer indication
 of which part of the stage failed - actual functionality or a style issue.

 \begin{figure}
\begin{center}
\includegraphics[width=4cm]{figures/cicd_coverage_buttons}
\end{center}
\caption{The buttons used to display pipeline status and test coverage}
\label{fig:cicd_coverage_buttons}
\end{figure}

% Describe deployment tasks
The continuous deployment tasks were both essentially the same but related to which branch was being
 committed to. As both \texttt{dev} and \texttt{master} can only be merged into instead of committed to, these tasks can
 be run only on merge commits to \texttt{dev} or \texttt{master}. The tasks run tests (to allow coverage reports for the branches)
 and then deploy to our live servers. The \texttt{dev} branch deploys to the staging zone, and
 \texttt{master} to our production site.

The fact that this is automated helps encourage rapid deployment, as the steps take some time, and are
 boring for people to do. This methodology takes out the human steps and means that the development team
 can focus on development. These rapid deployments also help by making it easy for the team to demonstrate
 and generate feedback from the public. This also allowed the client to continuously check our progress and
 give us feedback.


% consider how it helped us and how it could have been improved
The CI and CD processes helped us by keeping our code of high quality, preventing broken commits and reducing
 manual time spent doing menial tasks. However, it took a fairly large period of time to get working and to
 optimise into time chunks that were consistent with the goal of 10 minutes. It is arguable that the time
 could have been better spent working on the actual project itself. As a counterpoint to this, it could be
 argued that the CI and CD have saved the development team time in fixing bugs later on, and by ensuring code is more
 readable, reducing time wasted understanding the code.

 %==============================================================================
\section{Project Planning} %1-3 pages
\label{sec:planning}

After the project allocation day, GRN sent us a project brief containing
 more details about the project itself, as well as a list of functional
 requirements that the final product was expected to meet. The first
 task we needed to complete was a detailed requirements document, that would
 contain non-functional and any additional functional requirements. It was
 decided that the focus of the first month would be allocated to finalising
 the document, to assure that the client approved of the direction
 the team decided to take in working on the project.

As part of the requirements gathering process, the team developed four personas
 and an epic user story of how they envisioned the project being used. This was followed by
 writing down user stories centred around the following entities: team manager,
 sponsor, player, club supporter, rugby fan and international rugby fan. Doing
 this eased the process of identifying useful requirements and helped the team
 visualise what the final product should be offering potential users.

After the initial requirements were outlined, the team followed by designing
 wireframes of the web application. Initially, each team member delivered a
 pencil-drawn sketch of each expected application screen. Afterwards, the
 best design from each wireframe was merged into the main wireframes, drawn
 using draw.io. GRN warned us that the main wireframes would not offer useful
 guidance when thinking how the application should transition to a mobile format,
 so the team repeated the process to generate mobile wireframes.

Over the weeks dedicated to producing the requirements document, the team
 kept close contact with GRN, who have repeatedly offered helpful feedback
 and guidance regarding how the team can meet their expectations of the
 final product. This was in the form of a weekly progress email, and a monthly
 physical meeting, giving us some understanding of what the client found
 to be of most importance, and it helped us in the next stage in our
 project, which was prioritising the features of the application and separating
 them into achievable goals for each iteration.

There were several aspects to be taken into consideration while deciding
 on a concrete structure of the development process. We had to consider
 our lack of experience with the technologies to be used and the fact
 that the available time to work on the project would vary as the academic
 year progressed. Another thing to consider was the fact that due to the
 agile methodology we decided to follow, at the end of each iteration, the
 product needed to be functional. Thus, it was decided to settle on having
 smaller, more easily achievable goals in the beginning, which would
 incrementally become more complex as advanced. In the end, we settled
 on seven iterations, excluding the initial one focused on the requirements
 document.

The goals for each iteration become increasingly challenging, based on the
 expectation that the first one or two iterations would offer all team
 members some hands on experience with both Angular2 and Firebase. The
 goals for each iteration were the following:
\begin{itemize}
\item First iteration - login and sign up
\item Second iteration - dashboard and followed pages
\item Third iteration - post creation and management
\item Fourth iteration - view and edit profile, posts on comments
\item Fifth iteration - search for teams/clubs/players or fixtures
\item Sixth iteration - translation and additional tasks
\item seventh iteration - additional features, refinement, documentation
\end{itemize}

\begin{figure}[H]
\begin{center}
\includegraphics[width=15cm]{figures/planning_milestones}
\end{center}
\caption{Project milestones on GitLab}
\label{fig:planning_milestones}
\end{figure}


GitLab allowed us to create project milestones. Each of them had some
 issues associated with it, the issues consisting of the goals for the
 respective iteration split into several multiple smaller issues. The deadline
 of each iteration would also appear on each issue, becoming bright red as the
 date came nearer, to alert the person assigned to the task.

Throughout the project, our process was modified and improved by having team
 retrospectives around the end of each iteration. The format of the retrospectives
 consisted of every team member going through what they thought went well in
 our past iteration, what they thought went bad and what ideas they have to
 improve our workflow in future iterations. All these were discussed within
 the team and what the majority considered useful ideas were then included
 in the further development. The main points of each retrospective are
 documented on the wiki page of the project, which can be found on GitLab.

%==============================================================================
\section{Agile Methodology} %1-3 pages
\label{sec:agile}

% short description of agile development

Agile software development consists of a set of practices and methods derived
 from the principles described in the Agile Manifesto. Agile enforces a strong
 collaboration between the development team and the business stakeholders,
 self-organizing teams and frequent delivery of functioning
 software\cite{agile_overview}.

As agile development was highly recommended as part of our software development
 process, in the initial team meetings, we decided to use the Scrum
 methodology.

 % short description of SCRUM
 Scrum is an agile framework originally formalised for software development
 projects. Scrum defines several major roles within the framework: the Product
 Owner, the Scrum Master, the development team and the customer. The Product
 Owner should maintain communication with the stakeholders and create a prioritised
 list goals to be achieved. The Scrum Master keeps the team focused and leads team
 meetings\cite{scrum_overview}.

 % our agile development
 Within our team, Andrew was the Product Owner and Ruxandra was the Scrum Master.
  After the first few team meetings, where the focus was on finalising the initial
  requirements gathering, the focus was shifted to dividing the prioritised
  expected features. As a result of discussions between the Product Owner and
  the rest of the team, it was decided to split the work into seven sprints,
  each of them lasting two weeks. One sprint would end on Friday, and the next
  one would start the following Monday (Figure \ref{fig:agile_gantt}).


 It was decided that instead of daily scrums, the team would have two weekly
  meetings, one on Wednesdays, during the software development lab, and one on
  Fridays. The format of the meetings included everyone describing what they've
  worked on since the previous meeting and the work they plan on completing
  until the next meeting. Each team member also discussed any difficulties
  they encountered, how they overcame them or whether they feel that they
  need help with certain issues.

 The Scrum Master was tasked with closely observing these meetings. They had
  to determine whether the team could handle the workload of the respective
  sprint, if everyone in the team had something to work on and how to efficiently
  assign members to assist those who need help with something. The Scrum
  Master would also take into consideration any complaints or suggestions from
  the rest of the team, as well as initiate team retrospectives at the end of
  each sprint.

 As the project progressed, our scrum process underwent a few changes. Once
 the desired goals and the predicted timelines were better defined, it was
 decided that one meeting during the PSD lab would be sufficient, with any
 additional meetings to be arranged if deemed necessary. Meanwhile, the team
 would communicate daily through a group chat on Facebook messenger, in order
 to keep everyone updated with the general progress.

Towards the end of the first semester, our sprints needed quite a few
 adjustments, due to the increase in additional assignments from other
 courses, that everyone in the team had to complete. There was also a
 slight halt in communication during the winter break, as team members
 went home for the holidays. However, the additional time left at the
 end of our initial schedule allowed us to maintain the initial seven
 sprints and just shift everything by a few weeks.

As we became more comfortable working as a team and since at that
 point everyone was comfortable with the new technologies, team
 meetings became less structured and less formal. Along the way,
 the meetings became open discussions, while still making sure
 everyone is kept up to date. At the beginning of the second
 semester, our scrum framework transitioned into a less clearly
 defined agile framework, which still followed the main principles
 of the methodology. Sprints became of variable length (between
 one and three weeks) rather than a constant two weeks.

 \begin{figure}[H]
\begin{center}
\includegraphics[width=15cm]{figures/agile_gantt}
\end{center}
\caption{Gantt Chart for Development Process}
\label{fig:agile_gantt}
\end{figure}


At the beginning of the development process, the Product Owner split
 every goal into separate small issues. Over the course of the project,
 these were refined and split further. The members of the team were
 also encouraged to make small commits as often as possible and only
 merge one small issue at a time in a separate development branch,
 to avoid disrupting someone else's work by generating conflicts.
 Every merge request needed to be approved by at least one other
 team member following a code review, in order to maintain the quality
 of the product. These code reviews helped us to spot obvious mistakes,
 and to improve team understanding of how components worked.

At the end of each sprint, the product was fully or almost fully
 functional with respect to the features that needed to be completed
 up to that point. The end of most sprints coincided with the customer
 meetings. Thus, we were able to receive almost immediate feedback
 from both GRN and supervisors. This also offered us the opportunity
 to make additional improvements in the short time left between sprints.

To maintain the close communication with our client, as dictated by
 the agile methodology, the Product Owner would communicate weekly
 updates by email. The team was also in close contact with the
 development team of our client company, which was available
 for questions and pointed us towards useful resources. This
 was helpful in assuring that our version of the product would
 be as close as possible to the expectations of GRN, with respect
 to both design and implementation.

%==============================================================================
\section{Change Management and Version Control Methods}
\label{sec:changemgmt}

At the very start of the project, we used a shared Google Docs folder
 to exchange initial ideas, such as user stories, wireframes, etc.
 Of course, when we started coding, some version control
 repository had to be selected, and our choice was GitLab.
 The most attractive features of GitLab for us were
 Git, which we were mostly familiar with at this point, issue
 tracking, which enabled us to submit feature requests and bug reports
 with ease and built-in continuous integration.

After we gathered the requirements for the project and defined the general
 development timeline, the issue tracker had to be filled in. This was done by 
 the Product Owner mostly, but other team members were encouraged to report bugs and
 propose features too. GitLab allowed
 us to create the sprints (with the possibility of assigning a deadline
 to each of them) in the form of milestones, and afterwards, the feature 
 requests could be assigned to a
 particular sprint. Additional attributes, such as assignee (the person
 responsible for resolving the issue), labels (feature, customer request,
 hotfix, non-essential, etc.), weight (1-10; decided during team meetings
 and representing how long they would take in hours) could also be added.
 
GitLab also allowed us to prioritise our tickets using a number of metrics - 
 from the weight assigned to them to the labels. We gave \texttt{Hotfix} tickets
 the highest priority, as these represented bugs that existed on \texttt{master}. 
 Following these tickets were \texttt{Bugfix} tickets, which were bugs found on 
 \texttt{dev}. Then followed \texttt{Feature} and \texttt{Client Request} (
 issues raised at client meetings but not specified in the requirements document). 

For our branching model, we selected one called Gitflow\cite{gitflow}. At
 its core, there are two main branches - \texttt{dev} and \texttt{master}. The \texttt{master} branch
 is the production-ready branch, and it reflects the state of the latest
 production release. Meanwhile, the \texttt{dev} branch reflects the state of our
 latest development efforts. We usually merged the changes from \texttt{dev} into
 \texttt{master} once a week to ensure the fewest bugs possible, as most
 of the testing was done even before merging to the \texttt{dev} branch.

\begin{figure}[h]
\begin{center}
\includegraphics[width=15cm]{figures/changemgmt_merge_request}
\end{center}
\caption{Example of an approved merge request}
\label{fig:changemgmt_merge_request}
\end{figure}

In the Gitflow model, \texttt{dev} and \texttt{master} are not committed to. Therefore
 all the development has to be done in separate feature branches. 
 For each issue in the issue tracker, its assignee created a
 new branch via GitLab and started development in that branch. The 
 branches were named according to this formula of:
 
 \texttt{\$issueNumber - \$issueTitle}

Another of the useful features provided by GitLab is the ability to 
 provide templates for feature branches. This helped make sure that 
 feature requests and bugfixes had sufficient information in them to
 allow smooth development. For example, the feature template reminded
 the creator to add user stories and functional requirements. The bugfix
 template encouraged screenshots and steps for replicating the bug to be
 included. This meant that the reporter was not necessarily the person who
 resolved the ticket.
 
We encouraged every team member to commit often, as it encouraged transparency
 in the team by making completed work more visible. This policy also meant 
 that linting and tests were run more regularly, preventing a large buildup
 of tidying up work towards the end of the feature branch's lifetime.
 By using our CI system to test every commit (section \ref{sec:cicd}), we could
 catch bugs and linting issues early on, and fix them before they snowballed into
 bigger, more confusing issues. After the development of the feature had been completed, the assignee
 submitted a merge request for the feature branch to be merged into \texttt{dev}.
 
This merge request had to be approved by any other team member, following a
 code review. Code reviews allowed us to find mistakes in the source code not
 noticed by the developer, and therefore improve the quality of our project.
 If any mistakes were found, the reviewer could add some comments to the merge
 request, explaining what could be improved. 
 
Another blocker for merge request was merge conflicts. They occurred if Git could not
 figure out on its own which lines of the source code to add, keep or
 remove, in case that files in \texttt{dev} and in the feature branch differed
 greatly. The assignee was responsible for this, as they would know the most about their
 own code. This was also often in conjunction with the rest of the team, to ensure that the merge
 would not result in unexpected behaviour. In some cases, the merge process did result
 in breaking the build. This was quickly caught by the automated tests, and could
 thus be promptly resolved. 
 
Ultimately, for a branch to be merged into \texttt{dev}, it had to meet 3 criteria:
\begin{itemize}
\item[1] Passes tests and linting pipelines
\item[2] There are no merge conflicts
\item[3] The code passed a code review
\end{itemize}
On merge, the issue ticket was closed, the new code merged into \texttt{dev}
 and the feature branch being deleted. This meant that the repository was not 
 cluttered with old branches. With over 130 branches merged, this would have been
 a considerable number of branches

Encountering some issues with the workflow of GitLab was unavoidable.
 For example, if a feature branch was not being worked on for some time,
 it was getting behind \texttt{dev}, resulting in a lot of merge conflicts.
 The solution was either to resolve all of those manually or, if
 there was not much progress done on the feature branch - simply remove
 it and start over. Code reviews were not always performed carefully
 (or not at all, during crunch time), so problems were often encountered
 even on \texttt{dev}. On the other hand, the issue tracker did not
 cause any major issues; its level of convenience and robustness helped
 us get on the issues quickly, submit bug reports when encountered,
 and generally stay on top of the things.


%==============================================================================
\section{Conclusions} %1-2 pages
\label{sec:conclusion}

% Main Points:
% Co-ordinating teams
% Wasting time on setup instead of features
% Compromising on systems
% Client requirements affecting our choices
% Learning new technology is part of the industry

This software project was a valuable learning experience. Each and every one of
 us learned new technologies, in addition to a much-improved understanding
 of what working in a team over a long time involves. These lessons
 included: teamwork, organisation, time management and communication.

One of the biggest issues we first faced was coordinating ourselves as a
 new team of strangers. By using icebreaker exercises to get to know
 one another, and discussing our strengths and weaknesses, we got to
 understand where each member of the team was comfortable working. There
 were occasional difficulties in getting everyone to meetings and ensuring
 work was done on time, but by communicating well, we managed to share
 meeting information and help each other out on difficult issues.

We also discovered that while the project's architecture is important,
 it is also important to remain focused on the actual product. We
 spent a considerable amount of time at the start of the project setting
 up the VCS, branching strategy, GitLab and CI configuration. While these
 were all valuable to the project, it would have been worth taking a step
 back and asking "is this the best thing for the customer?". Finding a balance
 is important.

As GRN already uses certain technologies (such as Angular 2 and Firebase),
 we were asked to also use them in order to ensure compatibility. GRN also
 gave us feedback that differed from what we expected, resulting in us
 having to change our mindsets (particularly at the start of the year,
 before the team developed a clear vision for it). This taught us that
 while you may want to do things in your own way, the client is ultimately
 the decision-maker for many choices. By having regular contact with them
 allowed us to get feedback quickly and reduced the risk of a feature that
 would have to be completely scrapped.

Many of these technologies were new to the team (and to the industry),
 which meant that we had to learn them in a short period of time. This
 ate into development time but was also necessary to fulfil the client's
 needs. Learning these new technologies was interesting and challenging,
 but also came at the cost of time we could have spent on developing
 features and resulted in bugs that would not have existed if a more
 familiar framework had been used. Ultimately, learning to use new, different
 technologies is a fundamental part of being active in computer science,
 and the ability to adapt quickly to new ways of thinking will serve us well
 should we choose to go into industry.

The project gave us a chance to experience the world of software development
 first hand. We learnt about pitfalls to avoid in the future, but also how
 to work well in a team and put our education into practice. This experience
 will certainly be of great use to us in the future, from the technologies we
 learnt to the mistakes we made, to the late nights coding before client meetings.

%==============================================================================
\newpage
\bibliographystyle{plain} %1-2 pages
\bibliography{dissertation}
\end{document}
\grid
